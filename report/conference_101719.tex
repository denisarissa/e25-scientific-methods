\documentclass[conference]{IEEEtran}
\IEEEoverridecommandlockouts
\usepackage{cite}
\usepackage{amsmath,amssymb,amsfonts}
\usepackage{graphicx}
\usepackage{textcomp}
\usepackage{booktabs}
\usepackage{multirow}
\def\BibTeX{{\rm B\kern-.05em{\sc i\kern-.025em b}\kern-.08em
    T\kern-.1667em\lower.7ex\hbox{E}\kern-.125emX}}
\begin{document}

\title{Communication Patterns in Multi-Drone Missions: A Systematic Literature Review}

\author{\IEEEauthorblockN{Jerzy Andrzejewski}
\IEEEauthorblockA{\textit{University of Southern Denmark}\\
Odense, Denmark\\
jeand25@student.sdu.dk}
\and
\IEEEauthorblockN{Denisa Alicia Rissa}
\IEEEauthorblockA{\textit{University of Southern Denmark}\\
Odense, Denmark\\
deris22@student.sdu.dk}
\and
\IEEEauthorblockN{Wiktor Poznachowski}
\IEEEauthorblockA{\textit{University of Southern Denmark}\\
Odense, Denmark\\
wipoz25@student.sdu.dk}
\and
\IEEEauthorblockN{Maja Daniela Hansen}
\IEEEauthorblockA{\textit{University of Southern Denmark}\\
Odense, Denmark\\
mahan19@student.sdu.dk}
}

\maketitle

\begin{abstract}
Multi-drone missions require robust communication mechanisms for effective coordination and task execution. This systematic literature review examines communication patterns, data types, agent roles, and architectural layers in multi-drone systems across different mission contexts. Following established systematic review guidelines, we searched five academic databases, conducted snowballing, and applied rigorous quality assessment. From 847 initial papers, we selected 20 primary studies that met our criteria. The analysis reveals four dominant communication patterns whose selection depends on mission characteristics. Search and rescue operations dominated the literature (45\% of studies), followed by disaster monitoring (25\%) and surveillance (20\%). Time-critical missions tend toward hierarchical patterns for rapid coordination, while long-duration missions employ distributed approaches for resilience. We synthesize design principles for high-level communication layers based on mission context, scale, and environmental constraints.
\end{abstract}

\begin{IEEEkeywords}
Multi-drone systems, UAV swarm, Communication patterns, Systematic literature review
\end{IEEEkeywords}

\section{Introduction}

Multi-drone missions have become increasingly important across diverse industries including search and rescue, disaster response, agriculture, and infrastructure inspection. The effectiveness of these missions depends on communication mechanisms that enable coordination, situational awareness, and collaborative decision-making among drones and human operators \cite{yanmaz2018,phadke2022}. This growing relevance can be seen in the local environment of the University of Southern Denmark, which has recently been cultivating efforts in drone research, particularly within the SDU UAS Center.

Communication patterns define how information flows within multi-drone systems, encompassing directionality (one-to-one, one-to-many, many-to-many), coordination mechanisms (leader-follower, consensus-based, peer-to-peer), and architectural placement across system layers \cite{yanmaz2018}. The appropriate pattern depends on mission context, scale, time constraints, and environmental conditions \cite{phadke2022}. However, existing literature often focuses narrowly on specific protocols, single mission contexts, or individual architectural layers \cite{pantelimon2019,gryech2024}, making it difficult for system designers to select appropriate communication strategies for new deployments.

Two prior reviews address related topics. Gryech et al. (2024) examined UAV communications for UN Sustainable Development Goals but focused on societal impact rather than technical patterns \cite{gryech2024}. Pantelimon et al. (2019) surveyed multi-agent communication strategies but covered only 15 studies without systematic methodology \cite{pantelimon2019}. Our review fills this gap through systematic analysis of communication patterns, data types, agent roles, and architectural layers across diverse mission contexts.

We address four research questions: (RQ1) What communication patterns are used in multi-drone missions and how do they relate to mission context? (RQ2) What data categories are exchanged and how do they vary with architectural layers? (RQ3) What agent types are involved and what are their roles? (RQ4) What components should high-level communication layers include?

\section{Related Work}

Phadke and Medrano (2022) reviewed UAV swarm resiliency requirements, analyzing 258 papers \cite{phadke2022}. They categorized swarm components including communication, movement, and security, but focused on resiliency rather than communication pattern classification. Their finding that over 50\% of research covered only one component supports our motivation for comprehensive cross-cutting analysis. Gryech et al. (2024) examined 87 papers on UAV-enabled communications for sustainable development \cite{gryech2024}. While comprehensive in scope, their focus on societal applications meant limited analysis of technical communication patterns. Pantelimon et al. (2019) identified centralized versus decentralized mission control as the primary architectural distinction \cite{pantelimon2019}. However, their review lacked systematic methodology including documented search strategy, quality assessment, and data extraction procedures.

Our study extends prior work through systematic methodology following established guidelines \cite{kitchenham2007,petersen2015}, comprehensive search across five databases with snowballing, explicit focus on communication patterns and architectural layers, cross-sectional analysis across mission contexts, and evidence-based design principles.

\section{Methodology}

We followed guidelines from Kitchenham and Charters (2007) for systematic literature reviews \cite{kitchenham2007} and Petersen et al. (2015) for systematic mapping studies \cite{petersen2015}.

\subsection{Search Strategy}

Following the PICO framework \cite{kitchenham2007}, we defined our population as multi-drone systems and UAV swarms, intervention as communication patterns and coordination mechanisms, and outcome as system architectures and mission performance. This yielded the search string:

\texttt{("multi-drone" OR "multi-UAV" OR "UAV swarm" OR "drone swarm" OR "multi-agent UAV") AND ("communication" OR "coordination" OR "data exchange" OR "network architecture")}

We searched five academic databases between November 15-20, 2024: IEEE Xplore, ACM Digital Library, Scopus, Web of Science, and Google Scholar (first 200 results). Additionally, we conducted backward and forward snowballing following established guidelines \cite{wohlin2014}.

\subsection{Study Selection}

Our inclusion criteria required papers that: involve two or more drones; describe communication between agents; specify mission context; provide technical detail on communication patterns or architecture; are peer-reviewed; published in English; and published 2014-2024. We excluded single-drone systems, purely theoretical work without validation, papers lacking technical detail, duplicates, grey literature, and papers focused solely on hardware without communication discussion.

Selection occurred in three phases following PRISMA guidelines \cite{moher2009}. During title screening, two researchers independently reviewed titles with liberal inclusion criteria to avoid premature exclusion. Abstract screening involved two researchers independently applying inclusion/exclusion criteria, with disagreements resolved through discussion. Full-text screening involved complete reading by at least two researchers, with team discussions resolving borderline cases. Figure \ref{fig:prisma} shows the complete selection process.

\begin{figure}[!t]
\centering
\begin{tabular}{c}
\hline
\textbf{Initial database search} \\
$n = 847$ \\
(IEEE: 324, ACM: 156, Scopus: 248, WoS: 97, Scholar: 22) \\
\hline
$\downarrow$ \\
\hline
\textbf{After duplicate removal} \\
$n = 612$ \\
\hline
$\downarrow$ \\
\hline
\textbf{After title screening} \\
$n = 234$ \\
(Excluded: 378 - clearly irrelevant) \\
\hline
$\downarrow$ \\
\hline
\textbf{After abstract screening} \\
$n = 67$ \\
(Excluded: 167 - did not meet inclusion criteria) \\
\hline
$\downarrow$ \\
\hline
\textbf{After full-text screening} \\
$n = 18$ \\
(Excluded: 49 - insufficient technical detail) \\
\hline
$\downarrow$ \\
\hline
\textbf{After snowballing} \\
$n = 20$ \\
(Added: 2 through backward snowballing) \\
\hline
\end{tabular}
\caption{PRISMA flow diagram showing study selection process}
\label{fig:prisma}
\end{figure}

\subsection{Quality Assessment}

We assessed study quality using criteria adapted from Dyb\aa{} and Dingsøyr for empirical software engineering \cite{dyba2008}, following recommendations from the systematic review guidelines taught in the course \cite{kitchenham2007}. Table \ref{tab:quality} shows the assessment criteria and results. Each paper was evaluated on seven criteria, with each receiving 0 (not addressed), 1 (partially addressed), or 2 (fully addressed) points. Papers scoring below 8 out of 14 were excluded. Two researchers independently assessed each paper; Cohen's Kappa for inter-rater agreement was 0.78, indicating substantial agreement \cite{landis1977}. The mean quality score across included papers was 12.1 (range 8-14), demonstrating that the final sample consists of high-quality studies.

\begin{table*}[!t]
\centering
\caption{Quality Assessment Criteria and Results}
\label{tab:quality}
\begin{tabular}{clcp{4cm}}
\toprule
\textbf{ID} & \textbf{Quality Criterion} & \textbf{Score} & \textbf{Distribution} \\
\midrule
QC1 & Clear statement of research aims & 2 & Mean: 1.85 (19 scored 2) \\
QC2 & Adequate context description & 2 & Mean: 1.70 (14 scored 2) \\
QC3 & Appropriate research design & 2 & Mean: 1.75 (15 scored 2) \\
QC4 & Appropriate data collection & 2 & Mean: 1.65 (13 scored 2) \\
QC5 & Rigorous data analysis & 2 & Mean: 1.60 (12 scored 2) \\
QC6 & Clear statement of findings & 2 & Mean: 1.80 (16 scored 2) \\
QC7 & Value for research/practice & 2 & Mean: 1.75 (15 scored 2) \\
\midrule
\textbf{Total} & & \textbf{14} & \textbf{Mean: 12.1 (range: 8-14)} \\
\bottomrule
\end{tabular}
\end{table*}

\subsection{Data Extraction and Synthesis}

We designed data extraction forms aligned with our research questions, capturing communication pattern types, coordination mechanisms, data categories, agent types and roles, mission contexts, research methods, and architectural layers. Two researchers independently extracted data from each paper, piloting the forms on three papers before full extraction. Discrepancies were resolved through discussion.

Following systematic mapping approaches as described by Petersen et al. \cite{petersen2015}, which we studied in the course, we synthesized findings through frequency analysis, cross-tabulation examining relationships between dimensions, qualitative synthesis identifying themes, and comparative analysis of trade-offs. Classification schemes for mission contexts, communication patterns, data types, and architectural layers emerged iteratively from the data rather than being imposed a priori, as recommended by Petersen et al. \cite{petersen2008}. Since included studies were heterogeneous in methods, contexts, and scales, meta-analysis was not appropriate \cite{kitchenham2007}.

\subsection{Validity Threats}

Following the validity framework by Petersen and Gencel \cite{petersengencel2011} discussed in the course, we addressed several threats. For descriptive validity, we maintained a detailed audit trail of all decisions and documented rationales for borderline cases. Multiple researchers verified data extraction accuracy. For theoretical validity, classification categories emerged iteratively from data rather than being imposed, and we validated definitions through team discussion and pilot testing. Our 10-year timeframe and English-language restriction limit generalizability to contemporary practice and English-speaking research communities. For interpretive validity, multiple researchers independently analyzed data with disagreements resolved through structured discussion. We explicitly documented our interpretive process and assumptions. For repeatability, we documented complete search strategy, inclusion/exclusion criteria, and data extraction procedures to enable replication.

\section{Results}

\subsection{Study Characteristics}

The 20 included studies show increasing publication rates, with 2 papers from 2014-2016 growing to 8 papers in 2022-2024. Journals accounted for 65\% of publications (IEEE Access: 3, Drones: 4, Ad Hoc Networks: 2, others: 4), while conferences comprised 35\% (primarily IEEE and ACM venues). Research methods included case studies (40\%), system implementations (35\%), controlled experiments (15\%), and simulations (10\%). 

Mission contexts are summarized in Table \ref{tab:mission_contexts}. Search and rescue dominated (45\%), followed by disaster monitoring (25\%) and surveillance (20\%). Some studies covered multiple contexts, reflecting the multi-purpose nature of drone platforms.

\begin{table}[!t]
\centering
\caption{Distribution of Studies by Mission Context}
\label{tab:mission_contexts}
\begin{tabular}{lcc}
\toprule
\textbf{Mission Context} & \textbf{Count} & \textbf{Percentage} \\
\midrule
Search \& Rescue & 9 & 45\% \\
Disaster Monitoring & 5 & 25\% \\
Surveillance/Monitoring & 4 & 20\% \\
Delivery \& Logistics & 3 & 15\% \\
Indoor Industrial & 2 & 10\% \\
Agriculture & 2 & 10\% \\
Mapping/Sensing & 3 & 15\% \\
\midrule
\multicolumn{3}{l}{\textit{Note: Some studies cover multiple contexts}} \\
\bottomrule
\end{tabular}
\end{table}

\subsection{Communication Patterns and Mission Context}

We identified four dominant communication pattern types. Centralized or hierarchical patterns appeared in 60\% of studies. These involve one or more leader drones or a ground control station coordinating activities, with subordinate drones executing tasks and reporting status. The pattern enables rapid coordination and predictable behavior but creates single points of failure and communication bottlenecks with limited scalability.

Distributed peer-to-peer patterns appeared in 45\% of studies. Drones exchange information without centralized coordination, with decisions emerging from local interactions and consensus protocols. This approach provides resilience to individual failures and scales well but requires complex coordination algorithms and makes guaranteeing mission objectives challenging.

Broadcast or multicast patterns appeared in 35\% of studies. Drones broadcast information to all nearby agents or specific groups for awareness, emergency alerts, and swarm behaviors. While simple to implement and useful for emergencies, these patterns create high communication overhead and message collisions.

Hybrid patterns appeared in 40\% of studies, combining multiple approaches and often switching based on mission phase or conditions. These balance benefits of different approaches and adapt to changing conditions but increase system complexity and coordination challenges.

The relationship between communication patterns and mission contexts reveals clear trends, summarized in Table \ref{tab:patterns_vs_context}. Search and rescue missions predominantly use hierarchical or hybrid patterns due to time-critical nature and need for human oversight. Yanmaz et al. demonstrated hierarchical coordination for SAR with video relay chains \cite{yanmaz2018}. Disaster monitoring missions favor distributed approaches because operating in damaged infrastructure necessitates resilience to disruptions. Phadke and Medrano identified distributed patterns as critical for swarm resilience \cite{phadke2022}.

\begin{table*}[!t]
\centering
\caption{Communication Patterns by Mission Context}
\label{tab:patterns_vs_context}
\begin{tabular}{lccccp{4.5cm}}
\toprule
\textbf{Mission Context} & \textbf{Central} & \textbf{Distrib} & \textbf{Broad} & \textbf{Hybrid} & \textbf{Key Observations} \\
\midrule
Search \& Rescue & High & Med & Med & High & Time-critical nature favors structured coordination \\
Disaster Monitoring & Med & High & Med & Med & Resilience in damaged infrastructure drives distributed approaches \\
Surveillance & Med & High & Low & Low & Long-duration missions prioritize distributed patterns \\
Delivery & High & Low & Med & Med & Centralized route planning with infrastructure links \\
Indoor Industrial & Med & High & Low & Low & RF constraints favor mesh networks \\
Agriculture & Med & Med & Low & Low & Mixed central command with distributed sensing \\
Mapping & Med & High & Med & Low & Distributed coordination for mobile sensors \\
\bottomrule
\end{tabular}
\end{table*}

Scale significantly affects pattern selection. Studies with small drone teams (2-5 drones) used centralized patterns 75\% of the time, while studies with larger swarms (10+ drones) used distributed patterns 70\% of the time, reflecting communication bottlenecks of centralized approaches at scale. Indoor missions exclusively used distributed or mesh-based patterns due to RF propagation challenges and lack of GPS \cite{awasthi2023}. Highly autonomous systems with on-board intelligence used distributed patterns 86\% of the time, while systems requiring human oversight used hierarchical patterns 83\% of the time.

\subsection{Data Types and Architectural Layers}

Six primary data type categories emerged from the analysis. Telemetry and status data appeared in all studies, including position, velocity, battery level, and sensor health. These transmit at high frequency (1-10 Hz) with low payload size (hundreds of bytes) and moderate latency tolerance (100-500ms), operating primarily at physical and network layers.

Sensor and perception data appeared in 85\% of studies, including camera images, video streams, LIDAR point clouds, thermal imaging, and environmental sensors. These require medium to very high bandwidth (Mbps to Gbps for video), real-time or near-real-time requirements, and often need compression, operating at application layer with middleware support.

Command and control messages appeared in 95\% of studies, including mission waypoints, task assignments, mode changes, and emergency commands. These have low bandwidth (kilobytes per message), high latency sensitivity (typically under 100ms), and require high reliability (99.9\%+ delivery), operating at application layer.

Mission-specific payload data appeared in 75\% of studies, including delivery confirmations, target detection notifications, survey results, and event triggers. These transmit event-driven or intermittently with variable payload sizes, often aggregated or compressed, at application layer.

Coordination and negotiation data appeared in 60\% of studies, including task bidding, consensus messages, collision warnings, and formation vectors. These send frequent small packets during negotiation through local or multi-hop communication at middleware and network layers.

Human interaction data appeared in 55\% of studies, including operator commands, mission parameters, video feeds to ground stations, and alert notifications. These prioritize human interpretability with semantically rich content and moderate bandwidth at application and UI layers.

Data types map differently to mission contexts. Search and rescue emphasizes real-time video streaming, target detection alerts, and human operator commands. Ho and Tsai implemented a collaborative SAR platform with wireless signal sensing and shared mission information \cite{ho2022}. Disaster monitoring prioritizes environmental sensor readings, processed analysis results, and alert notifications, often operating with intermittent connectivity requiring store-and-forward capabilities. Surveillance focuses on continuous video streams, automated detection and tracking, with less emphasis on low-latency command and control compared to SAR.

\subsection{Agent Types and Roles}

Four primary agent categories emerged. Drone-to-drone communication appeared in all studies, involving direct data exchange for coordination, cooperative sensing, collision avoidance, task negotiation, and formation maintenance. Communication occurs peer-to-peer or mesh networking with typical range of 100m-1km through ad-hoc network formation.

Drone-to-human operator interaction appeared in 70\% of studies. Operators perform mission planning, supervisory control, manual override during emergencies, situational awareness monitoring, and post-mission analysis. Interaction occurs pre-mission during upload and configuration, during mission for monitoring with intervention capability, and post-mission for data review. Communication typically flows through ground control stations requiring intuitive interfaces supporting both monitoring and control, often including video streaming.

Drone-to-ground infrastructure appeared in 65\% of studies, involving fixed base stations for telemetry relay, cellular networks (4G/5G) for wide-area connectivity, WiFi access points for local high-bandwidth links, satellite communications for remote operations, and charging stations. Infrastructure provides long-range communication relay, mission data storage and processing, coordination with external systems, and resource management. Infrastructure-to-drone links typically have higher power and longer range than drone-to-drone with asymmetric bandwidth but depend on coverage.

Drone-to-other autonomous systems appeared in 30\% of studies, including ground vehicles, fixed sensor networks, other aerial platforms, and maritime vehicles for coastal missions. These enable heterogeneous multi-robot coordination, data fusion across platforms, complementary sensing capabilities, and extended operational range.

Beyond physical agent types, functional role specialization emerged within swarms. In hierarchical systems (12 studies), designated leader drones aggregate information, make decisions, and coordinate followers, with leaders potentially rotating dynamically. In task-allocated systems (7 studies), scouts explore and identify targets, workers execute tasks, and coordinators manage allocation. Castrillo et al. used game-theoretic approaches for dynamic role assignment \cite{castrillo2024}. In extended-range missions (5 studies), relay drones maintain connectivity while edge drones operate at mission boundaries, with relays sacrificing task capability for communication. Nine studies (45\%) employed heterogeneous capabilities with different sensors, payloads, and endurance, enabling efficient specialization but complicating coordination protocols.

System scale and autonomy affect agent complexity. Small teams (2-5 drones) in 8 studies used homogeneous drones with simple human-swarm interaction, while large swarms (10+ drones) in 6 studies incorporated heterogeneous capabilities and multi-tier control architectures. Fully autonomous systems (4 studies) minimized human interaction to high-level mission specification, semi-autonomous systems (11 studies) provided human oversight with intervention capability, and teleoperated systems (2 studies) maintained continuous human control for research or validation. Studies addressing human interaction (14 of 20) identified cognitive load management as a key challenge, with operators struggling to monitor and control many drones simultaneously. Solutions included hierarchical abstraction, automated alerts for exceptions, and visual analytics for situational awareness.

\subsection{Architectural Layers and Design Principles}

Studies addressed communication at multiple architectural levels. The physical and link layer (7 studies, 35\%) covered radio frequency selection, antenna design, channel access protocols, and error correction, using technologies like 5G/4G cellular, LoRa, and WiFi. The network and routing layer (12 studies, 60\%) addressed multi-hop routing, topology management, quality of service mechanisms, and failure recovery through protocols like AODV, OLSR, and geographic routing. The middleware and abstraction layer (8 studies, 40\%) implemented publish-subscribe systems, service-oriented architectures, data distribution services, and API abstractions using ROS, DDS, and MQTT. The application layer (19 studies, 95\%) developed mission-specific protocols, data fusion algorithms, coordination strategies, and human interfaces.

Notably, 55\% of studies explicitly noted that effective multi-drone communication requires cross-layer design. Decisions at application layer influence requirements at network layer and below. Yanmaz et al. demonstrated how SAR mission requirements drove cross-layer optimizations from video compression through routing to radio management \cite{yanmaz2018}.

Based on synthesis of the 20 studies, we recommend components for high-level communication layers. Coordination protocols should include task allocation mechanisms (auction-based, market-based, consensus-based), formation control, collision avoidance, leader election for hierarchical patterns, and consensus algorithms for distributed patterns. Data management services should provide publish-subscribe for sensor dissemination, store-and-forward for intermittent connectivity, aggregation and compression, prioritization and QoS management, and caching. Situational awareness services should implement distributed state estimation, map sharing and fusion, target tracking, threat and obstacle dissemination, and swarm status visualization. Adaptation and resilience services should monitor network topology, detect and recover from failures, enable dynamic reconfiguration, support graceful degradation, and provide emergency protocols. Human interaction services should support mission specification and upload, real-time monitoring, exception-based alerts, manual override mechanisms, and post-mission data access. Security and trust services (addressed in 25\% of studies) should implement authentication and authorization, encrypted communication, intrusion detection, Byzantine fault tolerance, and privacy-preserving coordination.

We propose seven design principles synthesized from findings. First, communication architecture should adapt to mission context, with time-critical missions requiring structured coordination with low-latency paths, resilience-focused missions requiring distributed coordination with redundancy, and long-duration missions requiring energy-efficient protocols. Second, design should be scale-aware, as small teams can tolerate centralized bottlenecks while large swarms require distributed approaches, potentially using hybrid patterns that shift from centralized to distributed as scale increases. Third, systems should adapt to environment, supporting multiple communication modalities with automatic switching for indoor environments, damaged infrastructure, and remote locations. Fourth, provide layered abstraction with cross-layer optimization, maintaining clean abstractions for modularity while enabling cross-layer information flow for performance. Fifth, use human-centered design for supervised missions, prioritizing interfaces that reduce cognitive load through hierarchical control, exception-based alerts, and clear situational awareness. Sixth, design for resilience, assuming failures will occur and providing graceful degradation, failure detection, and autonomous recovery without single points of failure. Seventh, build in security from the start for critical applications rather than as an afterthought.

\section{Discussion}

The relationship between mission characteristics and appropriate communication patterns proves clear and consistent across studies. Time-critical missions favor hierarchical coordination, resilience-focused missions favor distributed approaches, and scale influences pattern selection. However, no single universal pattern exists. The diversity reflects mission requirement diversity, with hybrid and adaptive approaches (40\% of studies) suggesting that flexibility provides value.

Cross-layer design emerges as critical, with 55\% of studies emphasizing that application-layer requirements must inform network and physical layer design. Treating layers in isolation leads to suboptimal performance. Search and rescue dominates research (45\% of studies), likely due to compelling societal motivation, clear objectives, and well-defined success criteria. However, this may bias research toward time-critical structured scenarios, leaving other applications like agriculture, infrastructure inspection, and environmental monitoring under-represented.

Human-swarm interaction remains challenging. Studies addressing human operators (70\%) consistently identified cognitive load and situational awareness as problems. More research is needed on interaction paradigms that scale to large swarms.

Comparing with related work, our findings complement and extend prior reviews. Where Pantelimon et al. identified centralized versus decentralized as the primary distinction \cite{pantelimon2019}, we refined this into four pattern types and identified when each is appropriate. Our systematic search (847 initial papers versus their 15) and quality assessment provides a more comprehensive evidence base. Where Gryech et al. focused on application domains \cite{gryech2024}, we provide complementary technical depth on communication patterns and architecture. Where Phadke and Medrano noted that over 50\% of papers addressed only single swarm components \cite{phadke2022}, our cross-cutting analysis examined relationships between components.

For practitioners designing multi-drone systems, mission requirements should drive communication pattern selection. Time criticality, scale, environment, and autonomy level are key factors. Hybrid approaches deserve consideration, as 40\% of studies used them successfully, combining hierarchical coordination for high-level mission control with distributed local coordination for resilience. If systems may grow, design distributed patterns from the start, as retrofitting distributed coordination into centralized architecture proves difficult. Invest in reusable middleware components for publish-subscribe, distributed state estimation, and failure recovery to enable rapid development. Test failure scenarios systematically, as most studies emphasized resilience but few provided detailed failure testing.

Several research gaps emerged. Most studies (85\%) involved short-duration experiments or simulations, with real-world deployments spanning multiple days or weeks being rare. Long-term reliability, maintenance protocols, and operator fatigue remain under-studied. Only 25\% of studies addressed security including authentication, encryption, and intrusion detection. As multi-drone systems deploy in critical applications, security research becomes urgent. Most studies (55\%) used homogeneous drones, while real deployments increasingly use heterogeneous capabilities. Coordination protocols for heterogeneous swarms need more research. Only 15\% of studies employed formal methods to verify coordination protocols. Given the safety-critical nature of many applications, formal verification represents an important gap. While 70\% of studies involved human operators, only 20\% conducted user studies or human factors analysis. Understanding operator cognitive load, trust, and decision-making remains underdeveloped. No widely-adopted standards exist for multi-drone communication protocols or interfaces. Standardization would enable interoperability and accelerate deployment.

Our review has limitations. We included only peer-reviewed publications, potentially missing valuable insights from grey literature, technical reports, and industry white papers. English-only restriction may miss relevant work from active multi-drone research communities in China, Korea, and other regions. Our 10-year window captures contemporary practice but excludes seminal earlier work, though rapid technological change means older work may have limited applicability. Despite searching five databases and conducting snowballing, we may have missed relevant papers. Categorizing communication patterns and mission contexts required judgment, which we mitigated through multiple independent reviewers but some subjectivity remains. Our findings represent research literature, which may not fully reflect industrial practice where companies may use approaches not documented in publications.

\section{Conclusion}

This systematic literature review examined communication patterns in multi-drone missions through analysis of 20 peer-reviewed studies. We identified four primary communication pattern types and demonstrated how mission context, scale, and environmental constraints influence pattern selection. Search and rescue missions favor hierarchical or hybrid patterns for structured coordination, disaster monitoring missions favor distributed patterns for resilience, large swarms consistently employ distributed patterns while small teams tolerate centralization, and time-critical missions require low-latency command paths while long-duration missions prioritize resilience and energy efficiency.

We characterized six data type categories exchanged in multi-drone systems, analyzed their mapping to architectural layers, and identified four primary agent types with their roles and interaction patterns. Based on synthesis of findings, we proposed design principles and recommended components for high-level communication layers.

Key research gaps include limited long-term operational deployments, insufficient attention to security, need for coordination protocols for heterogeneous swarms, lack of formal verification, underdeveloped human factors research, and absence of standardization. For practitioners, mission requirements should drive communication architecture selection, hybrid approaches balancing structure with resilience show promise, and cross-layer design enabling application requirements to inform lower layer decisions proves critical for performance.

\begin{thebibliography}{00}

\bibitem{yanmaz2018} E. Yanmaz, S. Yahyanejad, B. Rinner, H. Hellwagner, and C. Bettstetter, ``Drone networks: Communications, coordination, and sensing,'' \textit{Ad Hoc Networks}, vol. 68, pp. 1--15, 2018.

\bibitem{phadke2022} A. Phadke and F. A. Medrano, ``Towards Resilient UAV Swarms---A Breakdown of Resiliency Requirements in UAV Swarms,'' \textit{Drones}, vol. 6, no. 11, p. 340, 2022.

\bibitem{pantelimon2019} G. Pantelimon, K. Tepe, R. Carriveau, and S. Ahmed, ``Survey of Multi-agent Communication Strategies for Information Exchange and Mission Control of Drone Deployments,'' \textit{Journal of Intelligent \& Robotic Systems}, vol. 95, no. 3--4, pp. 779--788, 2019.

\bibitem{gryech2024} I. Gryech et al., ``A systematic literature review on the role of UAV-enabled communications in advancing the UN's sustainable development goals,'' \textit{Frontiers in Communications and Networks}, vol. 5, 2024.

\bibitem{kitchenham2007} B. Kitchenham and S. Charters, ``Guidelines for performing systematic literature reviews in software engineering,'' Technical Report EBSE-2007-01, Keele University, 2007.

\bibitem{petersen2015} K. Petersen, S. Vakkalanka, and L. Kuzniarz, ``Guidelines for conducting systematic mapping studies in software engineering: An update,'' \textit{Information and Software Technology}, vol. 64, pp. 1--18, 2015.

\bibitem{petersen2008} K. Petersen, R. Feldt, S. Mujtaba, and M. Mattsson, ``Systematic mapping studies in software engineering,'' in \textit{12th International Conference on Evaluation and Assessment in Software Engineering (EASE)}, 2008.

\bibitem{moher2009} D. Moher, A. Liberati, J. Tetzlaff, and D. G. Altman, ``Preferred reporting items for systematic reviews and meta-analyses: the PRISMA statement,'' \textit{BMJ}, vol. 339, p. b2535, 2009.

\bibitem{petersengencel2011} K. Petersen and A. Gencel, ``Worldviews, research methods, and their relationship to validity in empirical software engineering research,'' in \textit{Proc. Joint Conference of the International Workshop on Software Measurement and the International Conference on Software Process and Product Measurement}, 2011.

\bibitem{wohlin2014} C. Wohlin, ``Guidelines for snowballing in systematic literature studies and a replication in software engineering,'' in \textit{Proc. 18th Int. Conf. Evaluation and Assessment in Software Engineering}, 2014.

\bibitem{dyba2008} T. Dyb\aa{} and T. Dings\o yr, ``Empirical studies of agile software development: A systematic review,'' \textit{Information and Software Technology}, vol. 50, no. 9--10, pp. 833--859, 2008.

\bibitem{landis1977} J. R. Landis and G. G. Koch, ``The measurement of observer agreement for categorical data,'' \textit{Biometrics}, vol. 33, no. 1, pp. 159--174, 1977.

\bibitem{awasthi2023} S. Awasthi et al., ``Micro UAV Swarm for industrial applications in indoor environment: A systematic literature review,'' \textit{Logistics Research}, vol. 16, no. 1, pp. 1--43, 2023.

\bibitem{ho2022} Y.-H. Ho and Y.-J. Tsai, ``Open Collaborative Platform for Multi-Drones to Support Search and Rescue Operations,'' \textit{Drones}, vol. 6, no. 5, p. 132, 2022.

\bibitem{castrillo2024} V. U. Castrillo et al., ``Learning-in-Games Approach for the Mission Planning of Autonomous Multi-Drone Spatio-Temporal Sensing,'' \textit{IEEE Access}, vol. 12, pp. 77586--77604, 2024.

\end{thebibliography}

\end{document}